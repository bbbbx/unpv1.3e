\section{Sockets Introduction}

  \subsection{Socket Address Structures}

    大多数 socket 函数都需要一个指向 socket address structure 的指针作为参数。每个 protocol suite 都定义了它自己的 socket address structure。
    这些 structures 都以 \texttt{sockaddr\_} 开头并以一个对应着每个 protocol suite 的后缀来结尾。

    \subsubsection{Pv4 Socket Address Structure}

      一个 IPv4 的 socket address structure 就命名为 \texttt{sockaddr\_in},并被定义在 \texttt{netinet/in.h} 头文件中。

      \begin{minted}[
        frame=lines
      ]{c}
      struct sockaddr_in {
        uint8_t        sin_len;     /* structure 的 length(16) */
        sa_family_t    sin_family;  /* AF_INET */
        in_port_t      sin_port;    /* 16-bit TCP or UDP port number */
                                    /* network bit ordered */
        struct in_addr sin_addr;    /* 32-bit IPv4 address */
                                    /* network bit ordered */
        char           sin_zero[8]; /* 未使用 */
      };

      struct in_addr {
        in_addr_t      s_addr;  /* 32-bit IPv4 address */
                                /* network byte ordered */
      };
      \end{minted}

    \subsubsection{Generic Socket Address Structure}

      任何一个的 socket 函数都需要将传给它的 socket address structures 指针再进行处理。

      在 \texttt{<sys/socket.h>} 头文件中定义了一个 \textit{generic} socket address。

      \begin{minted}[
        frame=lines
      ]{c}
      struct sockaddr {
        uint8_t      sa_len;
        sa_family_t  sa_family;   /* address family: AF_xxx value */
        char         sa_data[14]; /* protocol-specific address */
      };
      \end{minted}

    \subsubsection{IPv6 Socket Address Structure}

      IPv6 的 socket address 被定义在 \texttt{<netinet/in.h>} 头文件中。

      \begin{minted}[
        frame=lines
      ]{c}
      #define SIN6_LEN /* required for compile-time tests */

      struct sockaddr_in6 {
        uint8_t         sin6_len;
        sa_family_t     sin6_family;
        in_port_t       sin6_port;

        uint32_t        sin6_flowinfo;  /* flow information, undefined */
        struct in6_addr sin6_addr;

        uint32_t        sin6_scope_id;  /* set of interfaces for a scope */
      };

      struct in6_addr {
        uint8_t    s6_addr[16];  /* 128-bit IPv6 address */
      };
        
      \end{minted}

    \subsubsection{New Generic Socket Address Structure}

      IPv6 socket API 定义了一个新的 generic socket address structure,用来克服 \texttt{struct sockaddr} 的缺点。
      \texttt{sockaddr\_storage} structure 被定义在 \texttt{<netinet/in.h>} 头文件中。

      \begin{minted}[
        frame=lines
      ]{c}
      struct sockaddr_storage {
        uint8_t      ss_len;
        sa_family_t  ss_family;
      };
      \end{minted}

    \subsubsection{Comparison of Socket Address Structures}

      本书会遇到 5 种 socket address structure:IPv4、IPv6、Unix domain、datalink 和 storage。

  \subsection{Value-Result Arguments}

    有一种参数被称为 \textit{value-result} argument。
    
    函数 \texttt{accept}、\texttt{recvform}、\texttt{getsockname} 和 \texttt{getpeername}。这四个函数的两个参数是:
    指向 socket address structure 的指针和一个指向 integer 的指针,该 integer 包含了该 structure 的 size。例如:

    \begin{minted}[
      frame=lines
    ]{c}
    struct sockaddr_un cli;    /* Unix domain */
    socklen_t          len;

    len = sizeof(cli);            /* len 是一个 value */
    getpeername(unixfd, (SA *)&cli, &len);
    /* len 的 value 可能会被改变 */
    \end{minted}

    该 structure 的 size 在函数被调用时是一个 \textit{value},而当函数返回时,size 就变成了 \textit{result}。这种类型就参数就称为 \textit{value-result} argument。

  \subsection{Byte Ordering Functions}


    什么是 \textit{little-endian} byte order?什么是 \textit{big-endian} byte order?

    我们把特定的 system 使用的 blyutdee:ordering 称为 \textit{host byte order}。大多数的 host byte order 都是 little-endian byte order。

    因为 networking protocols 需要指定一个 \textit{network byte order},所有,作为 network programmers 就必须要能够处理这些 byte ordering 的不同之处。
    而 Internet protocols 使用的是 big-endian byte ordering。

    这两种 bytes 之间可以使用下面四个函数进行转换。

    \begin{minted}[
      frame=lines
    ]{c}
    #include <netinet/in.h>

    uint16_t htons(uint16_t host16bitvalue);
    uint32_t htons(uint32_t host32bitvalue);

    uint16_t ntohs(uint16_t net16bitvalue);
    uint32_t ntohs(uint32_t net32bitvalue);
    \end{minted}

    其中 h 代表 host、n 代表 network、s 代表 short、l 代表 long。

  \subsection{\texttt{inet\_aton}、\texttt{inet\_addr} 和 \texttt{inet\_ntoa} 函数}

    这些函数可用于将 Internet addresses 在 ASCII string 和 network byte ordered binary values 之间转换。

    \begin{minted}[
      frame=lines
    ]{c}
    #include <arpa/inet.h>

    int inet_aton(const char *strptr, struct in_addr *addrptr);
    
    in_addr_t inet_addr(const char *strptr);

    char *inet_ntoa(struct in_addr_t inaddr);
    \end{minted}

    现在,人们都反对使用 \texttt{inet\_addr} 函数,而用函数 \texttt{inet\_aton} 来代替。更好的办法是使用 \texttt{inet\_pton},它对 IPv4 和 IPv6 都可以处理。

  \subsection{\texttt{inet\_pton} 和 \texttt{inet\_ntop} 函数}

    这两个函数比较新,对 IPv4 和 IPv6 的地址都可以处理。其中字母 “p” 表示 presentation,“n” 表示 numeric。
    而一个 address 的 presentation 格式通常是一个 ASCII string。

    \begin{minted}[
      frame=lines
    ]{c}
    #include <arpa/inet.h>

    int inet_pton(int family, const char *strptr, void *addrptr);

    const char *inet_ntop(int family, const void *addrptr, char *strptr, size_t len);
    \end{minted}

    其中的 \texttt{family} 可以是 \texttt{AF\_INET} 或 \texttt{AF\_INET6}。而 \texttt{len} 参数是 destination 的 size。
    为了有助于指定这个 size,\texttt{<netinet/in.h>} 头文件定义了下列两个 definitions:

    \begin{minted}[
      frame=lines
    ]{c}
    #define INET_ADDRSTRLEN     16   /* for IPv4 dotted-decimal */
    #define INET6_ADDRSTRLEN    46   /* for IPv6 hex string */
    \end{minted}
