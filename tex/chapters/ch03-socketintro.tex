\section{Sockets Introduction}

  \subsection{Socket Address Structures}

    大多数 socket 函数都需要一个指向 socket address structure 的指针作为参数。每个 protocol suite 都定义了它自己的 socket address structure。
    这些 structures 都以 \texttt{sockaddr\_} 开头并以一个对应着每个 protocol suite 的后缀来结尾。

    \subsubsection{Pv4 Socket Address Structure}

      一个 IPv4 的 socket address structure 就命名为 \texttt{sockaddr\_in},并被定义在 \texttt{netinet/in.h} 头文件中。

      \begin{minted}[
        frame=lines
      ]{c}
      struct sockaddr_in {
        uint8_t        sin_len;     /* structure 的 length(16) */
        sa_family_t    sin_family;  /* AF_INET */
        in_port_t      sin_port;    /* 16-bit TCP or UDP port number */
                                    /* network bit ordered */
        struct in_addr sin_addr;    /* 32-bit IPv4 address */
                                    /* network bit ordered */
        char           sin_zero[8]; /* 未使用 */
      };

      struct in_addr {
        in_addr_t      s_addr;  /* 32-bit IPv4 address */
                                /* network byte ordered */
      };
      \end{minted}

    \subsubsection{Generic Socket Address Structure}

      任何一个的 socket 函数都需要将传给它的 socket address structures 指针再进行处理。

      在 \texttt{<sys/socket.h>} 头文件中定义了一个 \textit{generic} socket address。

      \begin{minted}[
        frame=lines
      ]{c}
      struct sockaddr {
        uint8_t      sa_len;
        sa_family_t  sa_family;   /* address family: AF_xxx value */
        char         sa_data[14]; /* protocol-specific address */
      };
      \end{minted}

    \subsubsection{IPv6 Socket Address Structure}

      IPv6 的 socket address 被定义在 \texttt{<netinet/in.h>} 头文件中。

      \begin{minted}[
        frame=lines
      ]{c}
      #define SIN6_LEN /* required for compile-time tests */

      struct sockaddr_in6 {
        uint8_t         sin6_len;
        sa_family_t     sin6_family;
        in_port_t       sin6_port;

        uint32_t        sin6_flowinfo;  /* flow information, undefined */
        struct in6_addr sin6_addr;

        uint32_t        sin6_scope_id;  /* set of interfaces for a scope */
      };

      struct in6_addr {
        uint8_t    s6_addr[16];  /* 128-bit IPv6 address */
      };
        
      \end{minted}

    \subsubsection{New Generic Socket Address Structure}

      IPv6 socket API 定义了一个新的 generic socket address structure,用来克服 \texttt{struct sockaddr} 的缺点。
      \texttt{sockaddr\_storage} structure 被定义在 \texttt{<netinet/in.h>} 头文件中。

      \begin{minted}[
        frame=lines
      ]{c}
      struct sockaddr_storage {
        uint8_t      ss_len;
        sa_family_t  ss_family;
      };
      \end{minted}

    \subsubsection{Comparison of Socket Address Structures}

      本书会遇到 5 种 socket address structure:IPv4、IPv6、Unix domain、datalink 和 storage。