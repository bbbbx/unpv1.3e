\section{Elementary TCP Sockets}

  \subsection{\texttt{socket} 函数}

    要做 network I/O,进程第一件要做的事就是调用 \texttt{socket} 函数,用来指定 communication protocol 的类型(使用 IPv4 的 TCP、使用 IPv6 的 UDP、Unix domain stream protocol 等)。

    \begin{minted}[
      frame=lines
    ]{c}
    #include <sys/socket.h>

    int socket(int family, int type, int protocol);
    \end{minted}

  \subsection{\texttt{connect} 函数}
  

    \texttt{connect} 函数是被一个 TCP client 用来 establish 一个和 TCP server 的 connection 的。

    \begin{minted}[
      frame=lines
    ]{c}
    #include <sys/socket.h>

    int connect(int sockfd, const struct sockaddr *servaddr, socklen_t addrlen);
    \end{minted}

  \subsection{\texttt{bind} 函数}

    \texttt{bind} 函数用来将一个 local protocol address 赋值给一个 socket。

    \begin{minted}[
      frame=lines
    ]{c}
    #include <sys/socket.h>

    int bind(int sockfd, const struct sockaddr *myaddr, socklen_t addrlen);
    \end{minted}

  \subsection{\texttt{listen} 函数}

    \texttt{listen} 函数由一个 TCP server 来调用,该函数会做下列的两个 actions:

    \begin{enumerate}
      \item 当 \texttt{socket} 函数创建一个 socket 的时候,该 socket 是一个 active socket,而 \texttt{listen} 函数会将 unconnected socket 转换为一个 passive socket,
        会将 socket 从 CLOSE state 变为 LISTEN state。
      \item 该函数的第二个参数用来指定该 socket 的最大连接数。
    \end{enumerate}

    \begin{minted}[frame=lines]{c}
    #include <sys/socket.h>

    int listen(int sockfd, int backlog);
    \end{minted}

  \subsection{\texttt{accept} 函数}
  
    \texttt{accept} 函数由 TCP server 来调用,它返回在 completed connection queue 队头的下一个 completed connection。

    \begin{minted}[frame=lines]{c}
    #include <sys/socket.h>

    int listen(int sockfd, struct sockaddr *cliaddr, socklen_t *addrlen);
    \end{minted}

    注意,\texttt{addrlen} 是一个 value-result argument(Section 3.3)。

  \subsection{\texttt{close} 函数}

    Unix 通用的 \texttt{close} 函数用来 close 一个 socket 和 terminate 一个 TCP connection。

    \begin{minted}[frame=lines]{c}
    #include <sys/socket.h>

    int close(int sockfd);
    \end{minted}
    
    \subsubsection{Descriptor Reference Counts}

      \texttt{close} 函数只是将该 descriptor 的 reference count 减一。

  \subsection{\texttt{getsockname} 和 \texttt{getpeername} 函数}

    \texttt{getsockname} 函数用来返回一个 socket 的 local protocol address,\texttt{getpeername} 用来返回一个 socket 的 foreign protocol address。

    \begin{minted}[frame=lines]{c}
    #include <sys/socket.h>

    int getsockname(int sockfd, struct sockaddr *localaddr, socklen_t *addrlen);
      
    int getpeername(int sockfd, struct sockaddr *peeraddr, socklen_t *addrlen);
    \end{minted}

    注意,这两个函数的最后一个参数都是 value-result argument。也就是说,这两个函数都是用来填充 \texttt{localaddr} 或 \texttt{peeraddr} 指向的 socket address structure 的。
